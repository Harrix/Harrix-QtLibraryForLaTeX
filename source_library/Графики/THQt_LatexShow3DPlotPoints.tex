\textbf{Входные параметры:}

VMHL\_VectorX --- указатель на вектор координат X точек;
 
VMHL\_VectorY --- указатель на вектор координат Y точек;
 
VMHL\_VectorZ --- указатель на вектор координат Z точек;
 
VMHL\_N --- количество точек;
 
TitleChart --- заголовок графика;
 
NameVectorX --- название оси Ox. В формате: [обозначение], [расшифровка]. Например: u, Вероятность выбора;
 
NameVectorY --- название оси Oy. В формате: [обозначение], [расшифровка]. Например: q, Количество абрикосов;
 
NameVectorZ --- название оси Oz. В формате: [обозначение], [расшифровка]. Например: z, Вероятность;
 
Label --- label для графика;
 
ColorMap --- какой раскраски будет график. Возможны значения: mathcad, matlab, hot или тот, что вы хотите использовать. Рекомендуется mathcad.
 
ForNormalSize --- нормальный размер графика (на всю ширину), или для маленького размера график создается.
	
\textbf{Возвращаемое значение:}

Строка с Latex кодами с выводимым графиком.