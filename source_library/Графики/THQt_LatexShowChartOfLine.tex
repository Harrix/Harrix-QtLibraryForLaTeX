\textbf{Входные параметры:}

VMHL\_VectorX --- указатель на вектор координат X точек;
 
VMHL\_VectorY --- указатель на вектор координат Y точек;
 
VMHL\_N --- количество точек;
 
TitleChart --- заголовок графика;
 
NameVectorX --- название оси Ox. В формате: [обозначение], [расшифровка]. Например: u, Вероятность выбора;
 
NameVectorY --- название оси Oy. В формате: [обозначение], [расшифровка]. Например: q, Количество абрикосов;
 
NameLine --- название первого графика (для легенды);
 
Label --- label для графика
 
ShowLine --- показывать ли линию;
 
ShowPoints --- показывать ли точки;
 
ShowArea --- показывать ли закрашенную область под кривой;
 
ShowSpecPoints --- показывать ли специальные точки;
 
RedLine --- рисовать ли красную линию, или синюю;
 
ForNormalSize --- нормальный размер графика (на всю ширину), или для маленького размера график создается.
	
\textbf{Возвращаемое значение:}

Строка с Latex кодами с выводимым графиком.