\documentclass[a4paper,12pt]{article}

\input{packages}
\input{styles}

\title{HarrixQtLibraryForLaTeX v.1.32}
\author{А.\,Б. Сергиенко}
\date{\today}


\begin{document}

\input{names}

\maketitle

\begin{abstract}
Библиотека HarrixQtLibraryForLaTeX --- библиотека для отображения различных данных в LaTeX файлах.
\end{abstract}

\tableofcontents

\newpage

\section{Введение}

Библиотека HarrixQtLibraryForLaTeX --- это библиотека для отображения различных данных в LaTeX файлах.

Последнюю версию документа можно найти по адресу:

\href{https://github.com/Harrix/HarrixQtLibraryForLaTeX}{https://github.com/Harrix/HarrixQtLibraryForLaTeX}

Об установке библиотеки можно прочитать тут:

\href{http://blog.harrix.org/?p=1164}{http://blog.harrix.org/?p=1164}

С автором можно связаться по адресу \href{mailto:sergienkoanton@mail.ru}{sergienkoanton@mail.ru} или  \href{http://vk.com/harrix}{http://vk.com/harrix}.

Сайт автора, где публикуются последние новости: \href{http://blog.harrix.org/}{http://blog.harrix.org/}, а проекты располагаются по адресу \href{http://harrix.org/}{http://harrix.org/}.

%%%%%%%%%%%%%%%%%%%%%%%%%%%%%%%%%%%%%%%%%%%%%%%%%%%%%%%%%% ВСТАВЛЯТЬ НИЖЕ
\newpage
\section{Список функций}\label{section_listfunctions}
\textbf{Главные загрузочные функции}
\begin{enumerate}

\item \textbf{\hyperref[HQt_LatexBegin]{HQt\_LatexBegin}} --- Возвращает начало для полноценного Latex файла для шаблона https://github.com/Harrix/HarrixLaTeXDocumentTemplate

\item \textbf{\hyperref[HQt_LatexBeginArticle]{HQt\_LatexBeginArticle}} --- Возвращает начало для полноценного Latex файла в виде статьи для шаблона https://github.com/Harrix/HarrixLaTeXDocumentTemplate.

\item \textbf{\hyperref[HQt_LatexBeginArticleWithPgfplots]{HQt\_LatexBeginArticleWithPgfplots}} --- Возвращает начало для полноценного Latex файла в виде статьи для шаблона https://github.com/Harrix/HarrixLaTeXDocumentTemplate с использованием графиков через пакет pgfplots.

\item \textbf{\hyperref[HQt_LatexBeginWithPgfplots]{HQt\_LatexBeginWithPgfplots}} --- Возвращает начало для полноценного Latex файла для шаблона https://github.com/Harrix/HarrixLaTeXDocumentTemplate с использованием графиков через пакет pgfplots.

\item \textbf{\hyperref[HQt_LatexEnd]{HQt\_LatexEnd}} --- Возвращает концовку для полноценного Latex файла для шаблона https://github.com/Harrix/HarrixLaTeXDocumentTemplate

\end{enumerate}

\textbf{Графики}
\begin{enumerate}

\item \textbf{\hyperref[HQt_LatexDrawLine]{HQt\_LatexDrawLine}} --- Функция возвращает строку с Latex кодом отрисовки линии по функции Function.

\item \textbf{\hyperref[THQt_LatexDraw3DPlot]{THQt\_LatexDraw3DPlot}} --- Функция возвращает строку с Latex кодом отрисовки 3D поверхности по функции Function.

\item \textbf{\hyperref[THQt_LatexShow3DPlot]{THQt\_LatexShow3DPlot}} --- Функция возвращает строку с выводом некоторого 3D графика в виде поверхности.

\item \textbf{\hyperref[THQt_LatexShow3DPlotPoints]{THQt\_LatexShow3DPlotPoints}} --- Функция возвращает строку с выводом некоторого 3D графика в виде множества точек.

\item \textbf{\hyperref[THQt_LatexShowBar]{THQt\_LatexShowBar}} --- Функция возвращает строку с выводом некоторого графика гистограммы с Latex кодами.

\item \textbf{\hyperref[THQt_LatexShowChartOfLine]{THQt\_LatexShowChartOfLine}} --- Функция возвращает строку с выводом некоторого графика по точкам с Latex кодами.

\item \textbf{\hyperref[THQt_LatexShowChartsOfLineFromMatrix]{THQt\_LatexShowChartsOfLineFromMatrix}} --- Функция возвращает строку с выводом графиков из матрицы по точкам с Latex кодами.

\item \textbf{\hyperref[THQt_LatexShowIndependentChartsOfLineFromMatrix]{THQt\_LatexShowIndependentChartsOfLineFromMatrix}} --- Функция возвращает строку с выводом графиков из матрицы по точкам с Latex кодами. Нечетные столбцы --- это значения координат X графиков. Следующие за ними четные столбцы --- соответствующие значения Y. То есть графики друг от друга независимы.

\item \textbf{\hyperref[THQt_LatexShowTwoChartsOfLine]{THQt\_LatexShowTwoChartsOfLine}} --- Функция возвращает строку с выводом некоторых двух графиков по точкам с Latex кодами.

\item \textbf{\hyperref[THQt_LatexShowTwoIndependentChartsOfLine]{THQt\_LatexShowTwoIndependentChartsOfLine}} --- Функция возвращает строку с выводом некоторых двух независимых графиков по точкам с Latex кодами.

\item \textbf{\hyperref[THQt_LatexShowTwoIndependentChartsOfPointsAndLine]{THQt\_LatexShowTwoIndependentChartsOfPointsAndLine}} --- Функция возвращает строку с выводом некоторого двух независимых графиков по точкам с Latex кодами. Один график выводится в виде точек, а второй в виде линии. Удобно для отображения регрессий.

\end{enumerate}

\textbf{Обработка текста}
\begin{enumerate}

\item \textbf{\hyperref[HQt_ForcedWordWrap]{HQt\_ForcedWordWrap}} --- Функция расставляет принудительные переносы в стиле Latex.

\item \textbf{\hyperref[HQt_LatexGreenText]{HQt\_LatexGreenText}} --- Функция возвращает строку с выводом зеленого текста.

\item \textbf{\hyperref[HQt_LatexRedText]{HQt\_LatexRedText}} --- Функция возвращает строку с выводом красного текста.

\item \textbf{\hyperref[HQt_TextForLatexToText]{HQt\_TextForLatexToText}} --- Функция обрабатывает строку String из переделки функции HQt\_TextToTextForLatex в нормальную строку. Еще удаляются знаки \$, которые обрамляют формулы.

\item \textbf{\hyperref[HQt_TextToTextForLatex]{HQt\_TextToTextForLatex}} --- Функция переводит текст в текст, который можно добавить в Latex код. В-первую очередь, это экранирование некоторых элементов.

\item \textbf{\hyperref[THQt_LatexNumberToText]{THQt\_LatexNumberToText}} --- Функция выводит число VMHL\_X в строку Latex, причем число выделено жирным.

\end{enumerate}

\textbf{Показ математических выражений}
\begin{enumerate}

\item \textbf{\hyperref[THQt_LatexShowMatrix]{THQt\_LatexShowMatrix}} --- Функция возвращает строку с выводом некоторой матрицы VMHL\_Matrix с Latex кодами.

\item \textbf{\hyperref[THQt_LatexShowMatrix2]{THQt\_LatexShowMatrix2}} --- Функция возвращает строку с выводом некоторой матрицы VMHL\_Matrix с Latex кодами.

\item \textbf{\hyperref[THQt_LatexShowVector]{THQt\_LatexShowVector}} --- Функция возвращает строку с выводом некоторого вектора VMHL\_Vector с Latex кодами.

\item \textbf{\hyperref[THQt_LatexShowVector2]{THQt\_LatexShowVector2}} --- Функция возвращает строку с выводом некоторого вектора VMH\_Vector с Latex кодами.

\item \textbf{\hyperref[THQt_LatexShowVectorT]{THQt\_LatexShowVectorT}} --- Функция возвращает строку с выводом некоторого вектора VMHL\_Vector в транспонированном виде с Latex кодами.

\end{enumerate}

\textbf{Составные изображения}
\begin{enumerate}

\item \textbf{\hyperref[HQt_LatexBeginCompositionFigure]{HQt\_LatexBeginCompositionFigure}} --- Функция возвращает строку с выводом начала рисунка, состоящего из нескольких рисунков или графиков.

\item \textbf{\hyperref[HQt_LatexBeginFigureInCompositionFigure]{HQt\_LatexBeginFigureInCompositionFigure}} --- Функция возвращает строку с Latex кодом при добавлении дополнительного рисунка или графика в рисунок, состоящего из нескольких рисунков.

\item \textbf{\hyperref[HQt_LatexEndCompositionFigure]{HQt\_LatexEndCompositionFigure}} --- Функция возвращает строку с выводом окончания рисунка, состоящего из нескольких рисунков или графиков.

\item \textbf{\hyperref[HQt_LatexEndFigureInCompositionFigure]{HQt\_LatexEndFigureInCompositionFigure}} --- Функция возвращает строку с Latex кодом после добавлении дополнительного рисунка или графика в рисунок, состоящего из нескольких рисунков.

\end{enumerate}

\textbf{Таблицы}
\begin{enumerate}

\item \textbf{\hyperref[HQt_LatexShowTable]{HQt\_LatexShowTable}} --- Функция возвращает строку с выводом таблицы.

\end{enumerate}

\textbf{Текст}
\begin{enumerate}

\item \textbf{\hyperref[HQt_LatexShowAlert]{HQt\_LatexShowAlert}} --- Функция возвращает строку с выводом некоторого предупреждения.

\item \textbf{\hyperref[HQt_LatexShowHr]{HQt\_LatexShowHr}} --- Функция возвращает строку с выводом горизонтальной линии.

\item \textbf{\hyperref[HQt_LatexShowSection]{HQt\_LatexShowSection}} --- Функция возвращает строку с выводом некоторой строки в виде заголовка.

\item \textbf{\hyperref[HQt_LatexShowSimpleText]{HQt\_LatexShowSimpleText}} --- Функция возвращает строку с выводом некоторой строки с Latex кодами без всякого излишества.

\item \textbf{\hyperref[HQt_LatexShowSubsection]{HQt\_LatexShowSubsection}} --- Функция возвращает строку с выводом некоторой строки в виде подзаголовка.

\item \textbf{\hyperref[HQt_LatexShowText]{HQt\_LatexShowText}} --- Функция возвращает строку с выводом некоторой строки с Latex кодами.

\item \textbf{\hyperref[THQt_LatexShowNumber]{THQt\_LatexShowNumber}} --- Функция возвращает строку с выводом некоторого числа VMHL\_X с Latex кодами.

\end{enumerate}


\newpage
\section{Функции}
\subsection{Главные загрузочные функции}

\subsubsection{HQt\_LatexBegin}\label{HQt_LatexBegin}

Возвращает начало для полноценного Latex файла для шаблона https://github.com/Harrix/HarrixLaTeXDocumentTemplate


\begin{lstlisting}[label=code_syntax_HQt_LatexBegin,caption=Синтаксис]
QString HQt_LatexBegin();
\end{lstlisting}

\textbf{Входные параметры:}

Отсутствует.

\textbf{Возвращаемое значение:}

Начало для полноценного Latex файла.


\subsubsection{HQt\_LatexBeginArticle}\label{HQt_LatexBeginArticle}

Возвращает начало для полноценного Latex файла в виде статьи для шаблона https://github.com/Harrix/HarrixLaTeXDocumentTemplate.


\begin{lstlisting}[label=code_syntax_HQt_LatexBeginArticle,caption=Синтаксис]
QString HQt_LatexBeginArticle();
\end{lstlisting}

\textbf{Входные параметры:}

Отсутствует.

\textbf{Возвращаемое значение:}

Начало для полноценного Latex файла в виде статьи.


\subsubsection{HQt\_LatexBeginArticleWithPgfplots}\label{HQt_LatexBeginArticleWithPgfplots}

Возвращает начало для полноценного Latex файла в виде статьи для шаблона https://github.com/Harrix/HarrixLaTeXDocumentTemplate с использованием графиков через пакет pgfplots.


\begin{lstlisting}[label=code_syntax_HQt_LatexBeginArticleWithPgfplots,caption=Синтаксис]
QString HQt_LatexBeginArticleWithPgfplots();
\end{lstlisting}

\textbf{Входные параметры:}

Отсутствует.

\textbf{Возвращаемое значение:}

Начало для полноценного Latex файла в виде статьи с использованием графиков через пакет pgfplots.


\subsubsection{HQt\_LatexBeginWithPgfplots}\label{HQt_LatexBeginWithPgfplots}

Возвращает начало для полноценного Latex файла для шаблона https://github.com/Harrix/HarrixLaTeXDocumentTemplate с использованием графиков через пакет pgfplots.


\begin{lstlisting}[label=code_syntax_HQt_LatexBeginWithPgfplots,caption=Синтаксис]
QString HQt_LatexBeginWithPgfplots();
\end{lstlisting}

\textbf{Входные параметры:}

Отсутствует.

\textbf{Возвращаемое значение:}

Начало для полноценного Latex файла с использованием графиков через пакет pgfplots.


\subsubsection{HQt\_LatexEnd}\label{HQt_LatexEnd}

Возвращает концовку для полноценного Latex файла для шаблона https://github.com/Harrix/HarrixLaTeXDocumentTemplate


\begin{lstlisting}[label=code_syntax_HQt_LatexEnd,caption=Синтаксис]
QString HQt_LatexEnd();
\end{lstlisting}

\textbf{Входные параметры:}

Отсутствует.

\textbf{Возвращаемое значение:}

Концовка для полноценного Latex файла.


\subsection{Графики}

\subsubsection{HQt\_LatexDrawLine}\label{HQt_LatexDrawLine}

Функция возвращает строку с Latex кодом отрисовки линии по функции Function.


\begin{lstlisting}[label=code_syntax_HQt_LatexDrawLine,caption=Синтаксис]
QString HQt_LatexDrawLine (double Left, double Right, double h, double (*Function)(double), QString TitleChart, QString NameVectorX, QString NameVectorY, QString NameLine, bool ShowLine, bool ShowPoints, bool ShowArea, bool ShowSpecPoints, bool RedLine);
QString HQt_LatexDrawLine (double Left, double Right, double h, double (*Function)(double), QString TitleChart, QString NameVectorX, QString NameVectorY, bool ShowLine, bool ShowPoints, bool ShowArea, bool ShowSpecPoints, bool RedLine);
QString HQt_LatexDrawLine (double Left, double Right, double h, double (*Function)(double), QString TitleChart, QString NameVectorX, QString NameVectorY, QString NameLine);
QString HQt_LatexDrawLine (double Left, double Right, double h, double (*Function)(double));
\end{lstlisting}

\textbf{Входные параметры:}
 
    Left --- левая граница области;
 
    Right --- правая граница области;
 
    h --- шаг, с которым надо рисовать график;
 
    Function --- указатель на вычисляемую функцию;
 
    TitleChart --- заголовок графика;
 
    NameVectorX --- название оси Ox. В формате: [обозначение], [расшифровка]. Например: u, Вероятность выбора;
 
    NameVectorY --- название оси Oy. В формате: [обозначение], [расшифровка]. Например: q, Количество абрикосов;
 
    NameLine --- название первого графика (для легенды);
 
    ShowLine --- показывать ли линию;
 
    ShowPoints --- показывать ли точки;
 
    ShowArea --- показывать ли закрашенную область под кривой;
 
    ShowSpecPoints --- показывать ли специальные точки;
 
    RedLine --- рисовать ли красную линию, или синюю.

\textbf{Возвращаемое значение:}

Строка с Latex кодом.


\subsubsection{THQt\_LatexDraw3DPlot}\label{THQt_LatexDraw3DPlot}

Функция возвращает строку с Latex кодом отрисовки 3D поверхности по функции Function.


\begin{lstlisting}[label=code_syntax_THQt_LatexDraw3DPlot,caption=Синтаксис]
QString THQt_LatexDraw3DPlot (double Left_X, double Right_X, double Left_Y, double Right_Y, int N, double (*Function)(double, double),  QString TitleChart, QString NameVectorX, QString NameVectorY, QString NameVectorZ, QString Label, QString ColorMap, TypeOf3DPlot Type, double Opacity, double AngleHorizontal, double AngleVertical, bool ColorBar, bool ForNormalSize);
QString THQt_LatexDraw3DPlot (double Left_X, double Right_X, double Left_Y, double Right_Y, int N, double (*Function)(double, double),  QString TitleChart, QString NameVectorX, QString NameVectorY, QString NameVectorZ, QString Label, QString ColorMap, TypeOf3DPlot Type, bool ColorBar, bool ForNormalSize);
QString THQt_LatexDraw3DPlot (double Left_X, double Right_X, double Left_Y, double Right_Y, int N, double (*Function)(double, double),  QString TitleChart, QString NameVectorX, QString NameVectorY, QString NameVectorZ, QString Label, QString ColorMap, TypeOf3DPlot Type, bool ColorBar);
QString THQt_LatexDraw3DPlot (double Left_X, double Right_X, double Left_Y, double Right_Y, int N, double (*Function)(double, double),  QString TitleChart, QString NameVectorX, QString NameVectorY, QString NameVectorZ, QString Label, QString ColorMap, TypeOf3DPlot Type);
QString THQt_LatexDraw3DPlot (double Left_X, double Right_X, double Left_Y, double Right_Y, int N, double (*Function)(double, double),  QString TitleChart, QString NameVectorX, QString NameVectorY, QString NameVectorZ, QString Label);
QString THQt_LatexDraw3DPlot (double Left_X, double Right_X, double Left_Y, double Right_Y, int N, double (*Function)(double, double));
QString THQt_LatexDraw3DPlot (double Left, double Right, int N, double (*Function)(double, double));
\end{lstlisting}

\textbf{Входные параметры:}

Left\_X --- левая граница по оси Ox;
 
    Right\_X --- правая граница по оси Ox;
 
    Left\_Y --- левая граница по оси Oy;
 
    Right\_Y --- правая граница по оси Oy;
 
    N --- сколько нужно построить точек по каждой оси. В итоге получим N*N точек;
 
    Function --- ссылка на отрисовываемую двумерную функцию;
 
    TitleChart --- заголовок графика;
 
    NameVectorX --- название оси Ox. В формате: [обозначение], [расшифровка]. Например: u, Вероятность выбора;
 
    NameVectorY --- название оси Oy. В формате: [обозначение], [расшифровка]. Например: q, Количество абрикосов;
 
    NameVectorZ --- название оси Oz. В формате: [обозначение], [расшифровка]. Например: z, Вероятность;
 
    Label --- label для графика;
 
    ColorMap --- какой раскраски будет график. Возможны значения: mathcad, matlab, hot или тот, что вы хотите использовать. Рекомендуется mathcad;
 
    Type --- тип графика. Возможные значения:
 
       Plot3D\_Points --- в виде точек,
 
       Plot3D\_Surface --- в виде поверхности с непрерывной заливкой,
 
       Plot3D\_SurfaceGrid --- в виде поверхности с сеточной заливкой,
 
       Plot3D\_TopView --- вид сверху;
 
    Opacity --- прозрачность графика. Может изменяться от 0 до 1;
 
    AngleHorizontal --- угол поворота графика по горизонтали в градусах от ---180 до 180. Рекомендуется 25;
 
    AngleVertical --- угол поворота графика по вертикали в градусах от ---180 до 180. Рекомендуется 30;
 
    ColorBar --- рисовать с графиком колонку с градациями цветов или нет;
 
    ForNormalSize --- нормальный размер графика (на всю ширину), или для маленького размера график создается.

\textbf{Возвращаемое значение:}

Строка с Latex кодом.


\subsubsection{THQt\_LatexShow3DPlot}\label{THQt_LatexShow3DPlot}

Функция возвращает строку с выводом некоторого 3D графика в виде поверхности.


\begin{lstlisting}[label=code_syntax_THQt_LatexShow3DPlot,caption=Синтаксис]
template <class T> QString THQt_LatexShow3DPlot (T *VMHL_VectorX, T *VMHL_VectorY, T **VMHL_VectorZ,  int VMHL_N,  int VMHL_M, QString TitleChart, QString NameVectorX, QString NameVectorY, QString NameVectorZ, QString Label, QString ColorMap, TypeOf3DPlot Type, double Opacity, double AngleHorizontal, double AngleVertical, bool ColorBar, bool ForNormalSize);
template <class T> QString THQt_LatexShow3DPlot (T *VMHL_VectorX, T *VMHL_VectorY, T **VMHL_VectorZ,  int VMHL_N,  int VMHL_M, QString TitleChart, QString NameVectorX, QString NameVectorY, QString NameVectorZ, QString Label, QString ColorMap, TypeOf3DPlot Type, bool ColorBar, bool ForNormalSize);
template <class T> QString THQt_LatexShow3DPlot (T *VMHL_VectorX, T *VMHL_VectorY, T **VMHL_VectorZ,  int VMHL_N,  int VMHL_M, QString TitleChart, QString NameVectorX, QString NameVectorY, QString NameVectorZ, QString Label, QString ColorMap, TypeOf3DPlot Type, bool ColorBar);
template <class T> QString THQt_LatexShow3DPlot (T *VMHL_VectorX, T *VMHL_VectorY, T **VMHL_VectorZ,  int VMHL_N,  int VMHL_M, QString TitleChart, QString NameVectorX, QString NameVectorY, QString NameVectorZ, QString Label, QString ColorMap, TypeOf3DPlot Type);
template <class T> QString THQt_LatexShow3DPlot (T *VMHL_VectorX, T *VMHL_VectorY, T **VMHL_VectorZ,  int VMHL_N,  int VMHL_M, QString TitleChart, QString NameVectorX, QString NameVectorY, QString NameVectorZ, QString Label);
template <class T> QString THQt_LatexShow3DPlot (T *VMHL_VectorX, T *VMHL_VectorY, T **VMHL_VectorZ,  int VMHL_N,  int VMHL_M);
\end{lstlisting}

\textbf{Входные параметры:}
 

 
    VMHL\_VectorX --- указатель на вектор значений координат X сетки точек. Количество элементов VMHL\_N;
 
    VMHL\_VectorY --- указатель на вектор значений координат Y сетки точек. Количество элементов VMHL\_M;
 
    VMHL\_VectorZ --- указатель на матрицу значений координат Z точек. Количество элементов VMHL\_NxVMHL\_M;
 
    VMHL\_N --- количество значений в сетке по оси Ox;
 
    VMHL\_M --- количество значений в сетке по оси Oy;
 
    TitleChart --- заголовок графика;
 
    NameVectorX --- название оси Ox. В формате: [обозначение], [расшифровка]. Например: u, Вероятность выбора;
 
    NameVectorY --- название оси Oy. В формате: [обозначение], [расшифровка]. Например: q, Количество абрикосов;
 
    NameVectorZ --- название оси Oz. В формате: [обозначение], [расшифровка]. Например: z, Вероятность;
 
    Label --- label для графика;
 
    ColorMap --- какой раскраски будет график. Возможны значения: mathcad, matlab, hot или тот, что вы хотите использовать. Рекомендуется mathcad;
 
    Type --- тип графика. Возможные значения:
 
       Plot3D\_Points --- в виде точек,
 
       Plot3D\_Surface --- в виде поверхности с непрерывной заливкой,
 
       Plot3D\_SurfaceGrid --- в виде поверхности с сеточной заливкой,
 
       Plot3D\_TopView --- вид сверху;
 
    Opacity --- прозначность графика. Может изменяться от 0 до 1;
 
    AngleHorizontal --- угол поворота графика по горизонтали в градусах от ---180 до 180. Рекомендуется 25;
 
    AngleVertical --- угол поворота графика по вертикали в градусах от ---180 до 180. Рекомендуется 30;
 
    ColorBar --- рисоватm с графиком колонку с градациями цветов или нет;
 
    ForNormalSize --- нормальный размер графика (на всю ширину), или для маленького размера график создается.
	
\textbf{Возвращаемое значение:}

Строка с Latex кодами с выводимым графиком.


\subsubsection{THQt\_LatexShow3DPlotPoints}\label{THQt_LatexShow3DPlotPoints}

Функция возвращает строку с выводом некоторого 3D графика в виде множества точек.


\begin{lstlisting}[label=code_syntax_THQt_LatexShow3DPlotPoints,caption=Синтаксис]
template <class T> QString THQt_LatexShow3DPlotPoints (T *VMHL_VectorX, T *VMHL_VectorY, T *VMHL_VectorZ,  int VMHL_N, QString TitleChart, QString NameVectorX, QString NameVectorY, QString NameVectorZ, QString Label, QString ColorMap, bool ForNormalSize);
template <class T> QString THQt_LatexShow3DPlotPoints (T *VMHL_VectorX, T *VMHL_VectorY, T *VMHL_VectorZ,  int VMHL_N, QString TitleChart, QString NameVectorX, QString NameVectorY, QString NameVectorZ, QString Label, bool ForNormalSize);
template <class T> QString THQt_LatexShow3DPlotPoints (T *VMHL_VectorX, T *VMHL_VectorY, T *VMHL_VectorZ,  int VMHL_N, QString TitleChart, QString NameVectorX, QString NameVectorY, QString NameVectorZ, QString Label);
template <class T> QString THQt_LatexShow3DPlotPoints (T *VMHL_VectorX, T *VMHL_VectorY, T *VMHL_VectorZ,  int VMHL_N);
\end{lstlisting}

\textbf{Входные параметры:}
 

VMHL\_VectorX --- указатель на вектор координат X точек;
 
    VMHL\_VectorY --- указатель на вектор координат Y точек;
 
    VMHL\_VectorZ --- указатель на вектор координат Z точек;
 
    VMHL\_N --- количество точек;
 
    TitleChart --- заголовок графика;
 
    NameVectorX --- название оси Ox. В формате: [обозначение], [расшифровка]. Например: u, Вероятность выбора;
 
    NameVectorY --- название оси Oy. В формате: [обозначение], [расшифровка]. Например: q, Количество абрикосов;
 
    NameVectorZ --- название оси Oz. В формате: [обозначение], [расшифровка]. Например: z, Вероятность;
 
    Label --- label для графика;
 
    ColorMap --- какой раскраски будет график. Возможны значения: mathcad, matlab, hot или тот, что вы хотите использовать. Рекомендуется mathcad.
 
    ForNormalSize --- нормальный размер графика (на всю ширину), или для маленького размера график создается.
	
\textbf{Возвращаемое значение:}

Строка с Latex кодами с выводимым графиком.


\subsubsection{THQt\_LatexShowBar}\label{THQt_LatexShowBar}

Функция возвращает строку с выводом некоторого графика гистограммы с Latex кодами.


\begin{lstlisting}[label=code_syntax_THQt_LatexShowBar,caption=Синтаксис]
template <class T> QString THQt_LatexShowBar (T *VMHL_Vector, int VMHL_N, QString TitleChart, QString *NameVectorX, QString NameVectorY, QString Label, bool ForNormalSize, bool MinZero);
template <class T> QString THQt_LatexShowBar (T *VMHL_Vector, int VMHL_N, QString TitleChart, QString *NameVectorX, QString NameVectorY, QString Label, bool ForNormalSize);
template <class T> QString THQt_LatexShowBar (T *VMHL_Vector, int VMHL_N, QString TitleChart, QString *NameVectorX, QString NameVectorY, QString Label);
template <class T> QString THQt_LatexShowBar (T *VMHL_Vector, int VMHL_N);
template <class T> QString THQt_LatexShowBar (T *VMHL_Vector, int VMHL_N, QString TitleChart, QStringList NameVectorX, QString NameVectorY, QString Label, bool ForNormalSize, bool MinZero);
template <class T> QString THQt_LatexShowBar (T *VMHL_Vector, int VMHL_N, QString TitleChart, QStringList NameVectorX, QString NameVectorY, QString Label, bool ForNormalSize);
template <class T> QString THQt_LatexShowBar (T *VMHL_Vector, int VMHL_N, QString TitleChart, QStringList NameVectorX, QString NameVectorY, QString Label);
\end{lstlisting}

\textbf{Входные параметры:}
 
    VMHL\_Vector --- указатель на вектор значений точек;
 
    VMHL\_N --- количество точек;
 
    TitleChart --- заголовок графика;
 
    NameVectorX --- название значений точек. Будут подписаны под каждым столбиком на оси Ox. Количество элементов VMHL\_N;
 
    NameVectorY --- название оси Oy. В формате: [обозначение], [расшифровка]. Например: q, Количество абрикосов;
 
    Label --- label для графика;
 
    ForNormalSize --- нормальный размер графика (на всю ширину), или для маленького размера график создается;
 
    MinZero --- гистограмму начинать с нуля (true) или с минимального значения среди VMHL\_Vector (false).
	
\textbf{Возвращаемое значение:}

Строка с Latex кодами с выводимым графиком.


\subsubsection{THQt\_LatexShowChartOfLine}\label{THQt_LatexShowChartOfLine}

Функция возвращает строку с выводом некоторого графика по точкам с Latex кодами.


\begin{lstlisting}[label=code_syntax_THQt_LatexShowChartOfLine,caption=Синтаксис]
template <class T> QString THQt_LatexShowChartOfLine (T *VMHL_VectorX,T *VMHL_VectorY, int VMHL_N, QString TitleChart, QString NameVectorX, QString NameVectorY, QString NameLine, QString Label, bool ShowLine, bool ShowPoints, bool ShowArea, bool ShowSpecPoints, bool RedLine, bool ForNormalSize);
template <class T> QString THQt_LatexShowChartOfLine (T *VMHL_VectorX,T *VMHL_VectorY, int VMHL_N, QString TitleChart, QString NameVectorX, QString NameVectorY, QString NameLine, QString Label, bool ShowLine, bool ShowPoints, bool ShowArea, bool ShowSpecPoints, bool RedLine);
template <class T> QString THQt_LatexShowChartOfLine (T *VMHL_VectorX,T *VMHL_VectorY, int VMHL_N, QString TitleChart, QString NameVectorX, QString NameVectorY, QString NameLine, QString Label, bool ShowLine, bool ShowPoints, bool ShowArea, bool ShowSpecPoints);
template <class T> QString THQt_LatexShowChartOfLine (T *VMHL_VectorX,T *VMHL_VectorY, int VMHL_N, QString TitleChart, QString NameVectorX, QString NameVectorY, QString NameLine, QString Label);
template <class T> QString THQt_LatexShowChartOfLine (T *VMHL_VectorX,T *VMHL_VectorY, int VMHL_N);
\end{lstlisting}

\textbf{Входные параметры:}

    VMHL\_VectorX --- указатель на вектор координат X точек;
 
    VMHL\_VectorY --- указатель на вектор координат Y точек;
 
    VMHL\_N --- количество точек;
 
    TitleChart --- заголовок графика;
 
    NameVectorX --- название оси Ox. В формате: [обозначение], [расшифровка]. Например: u, Вероятность выбора;
 
    NameVectorY --- название оси Oy. В формате: [обозначение], [расшифровка]. Например: q, Количество абрикосов;
 
    NameLine --- название первого графика (для легенды);
 
    Label --- label для графика
 
    ShowLine --- показывать ли линию;
 
    ShowPoints --- показывать ли точки;
 
    ShowArea --- показывать ли закрашенную область под кривой;
 
    ShowSpecPoints --- показывать ли специальные точки;
 
    RedLine --- рисовать ли красную линию, или синюю;
 
    ForNormalSize --- нормальный размер графика (на всю ширину), или для маленького размера график создается.
	
\textbf{Возвращаемое значение:}

Строка с Latex кодами с выводимым графиком.


\subsubsection{THQt\_LatexShowChartsOfLineFromMatrix}\label{THQt_LatexShowChartsOfLineFromMatrix}

Функция возвращает строку с выводом графиков из матрицы по точкам с Latex кодами.


\begin{lstlisting}[label=code_syntax_THQt_LatexShowChartsOfLineFromMatrix,caption=Синтаксис]
template <class T> QString THQt_LatexShowChartsOfLineFromMatrix (T **VMHL_MatrixXY,int VMHL_N,int VMHL_M, QString TitleChart, QString NameVectorX, QString NameVectorY,QString *NameLine, QString Label, bool ShowLine,bool ShowPoints,bool ShowArea,bool ShowSpecPoints, bool ForNormalSize, bool GrayStyle, bool SolidStyle, bool CircleStyle);
template <class T> QString THQt_LatexShowChartsOfLineFromMatrix (T **VMHL_MatrixXY,int VMHL_N,int VMHL_M, QString TitleChart, QString NameVectorX, QString NameVectorY,QString *NameLine, QString Label, bool ShowLine,bool ShowPoints,bool ShowArea,bool ShowSpecPoints, bool ForNormalSize);
template <class T> QString THQt_LatexShowChartsOfLineFromMatrix (T **VMHL_MatrixXY,int VMHL_N,int VMHL_M, QString TitleChart, QString NameVectorX, QString NameVectorY,QString *NameLine, QString Label, bool ShowLine,bool ShowPoints,bool ShowArea,bool ShowSpecPoints);
template <class T> QString THQt_LatexShowChartsOfLineFromMatrix (T **VMHL_MatrixXY,int VMHL_N,int VMHL_M, QString TitleChart, QString NameVectorX, QString NameVectorY,QString *NameLine, QString Label);
template <class T> QString THQt_LatexShowChartsOfLineFromMatrix (T **VMHL_MatrixXY,int VMHL_N,int VMHL_M);
\end{lstlisting}

\textbf{Входные параметры:}
 
VMHL\_MatrixXY --- указатель на матрицу значений X и Y графиков;
 
    VMHL\_N --- количество точек;
 
    VMHL\_M --- количество столбцов матрицы (1+количество графиков);
 
    TitleChart --- заголовок графика;
 
    NameVectorX --- название оси Ox. В формате: [обозначение], [расшифровка]. Например: u, Вероятность выбора;;
 
    NameVectorY --- название оси Oy. В формате: [обозначение], [расшифровка]. Например: q, Количество абрикосов;
 
    NameLine --- указатель на вектор названий графиков (для легенды) количество элементов VMHL\_M---1 (так как первый столбец --- это X значения);
 
    Label --- label для графика;
 
    ShowLine --- показывать ли линию;
 
    ShowPoints --- показывать ли точки;
 
    ShowArea --- показывать ли закрашенную область под кривой;
 
    ShowSpecPoints --- показывать ли специальные точки;
 
    ForNormalSize --- нормальный размер графика (на всю ширину) или для маленького размера график создается;
 
    GrayStyle --- серый стиль графиков;
 
    SolidStyle --- линии делать сплошными или разными по типу (точками, тире и др.);
 
    CircleStyle --- точки все делать кругляшками или нет.
	
\textbf{Возвращаемое значение:}

Строка с Latex кодами с выводимым графиком.


\subsubsection{THQt\_LatexShowIndependentChartsOfLineFromMatrix}\label{THQt_LatexShowIndependentChartsOfLineFromMatrix}

Функция возвращает строку с выводом графиков из матрицы по точкам с Latex кодами. Нечетные столбцы --- это значения координат X графиков. Следующие за ними четные столбцы --- соответствующие значения Y. То есть графики друг от друга независимы.


\begin{lstlisting}[label=code_syntax_THQt_LatexShowIndependentChartsOfLineFromMatrix,caption=Синтаксис]
template <class T> QString THQt_LatexShowIndependentChartsOfLineFromMatrix (T **VMHL_MatrixXY,int *VMHL_N_EveryCol,int VMHL_M, QString TitleChart, QString NameVectorX, QString NameVectorY,QString *NameLine, QString Label, bool ShowLine,bool ShowPoints,bool ShowArea,bool ShowSpecPoints, bool ForNormalSize, bool GrayStyle, bool SolidStyle, bool CircleStyle);
template <class T> QString THQt_LatexShowIndependentChartsOfLineFromMatrix (T **VMHL_MatrixXY,int *VMHL_N_EveryCol,int VMHL_M, QString TitleChart, QString NameVectorX, QString NameVectorY,QString *NameLine, QString Label, bool ShowLine,bool ShowPoints,bool ShowArea,bool ShowSpecPoints, bool ForNormalSize);
template <class T> QString THQt_LatexShowIndependentChartsOfLineFromMatrix (T **VMHL_MatrixXY,int *VMHL_N_EveryCol,int VMHL_M, QString TitleChart, QString NameVectorX, QString NameVectorY,QString *NameLine, QString Label, bool ShowLine,bool ShowPoints,bool ShowArea,bool ShowSpecPoints);
template <class T> QString THQt_LatexShowIndependentChartsOfLineFromMatrix (T **VMHL_MatrixXY,int *VMHL_N_EveryCol,int VMHL_M, QString TitleChart, QString NameVectorX, QString NameVectorY,QString *NameLine, QString Label);
template <class T> QString THQt_LatexShowIndependentChartsOfLineFromMatrix (T **VMHL_MatrixXY,int *VMHL_N_EveryCol,int VMHL_M);
template <class T> QString THQt_LatexShowIndependentChartsOfLineFromMatrix (T **VMHL_MatrixXY,int VMHL_N,int VMHL_M, QString TitleChart, QString NameVectorX, QString NameVectorY,QString *NameLine, QString Label, bool ShowLine,bool ShowPoints,bool ShowArea,bool ShowSpecPoints, bool ForNormalSize, bool GrayStyle, bool SolidStyle, bool CircleStyle);
template <class T> QString THQt_LatexShowIndependentChartsOfLineFromMatrix (T **VMHL_MatrixXY,int VMHL_N,int VMHL_M, QString TitleChart, QString NameVectorX, QString NameVectorY,QString *NameLine, QString Label, bool ShowLine,bool ShowPoints,bool ShowArea,bool ShowSpecPoints, bool ForNormalSize);
template <class T> QString THQt_LatexShowIndependentChartsOfLineFromMatrix (T **VMHL_MatrixXY,int VMHL_N,int VMHL_M, QString TitleChart, QString NameVectorX, QString NameVectorY,QString *NameLine, QString Label, bool ShowLine,bool ShowPoints,bool ShowArea,bool ShowSpecPoints);
template <class T> QString THQt_LatexShowIndependentChartsOfLineFromMatrix (T **VMHL_MatrixXY,int VMHL_N,int VMHL_M, QString TitleChart, QString NameVectorX, QString NameVectorY,QString *NameLine, QString Label);
template <class T> QString THQt_LatexShowIndependentChartsOfLineFromMatrix (T **VMHL_MatrixXY,int VMHL_N,int VMHL_M);
\end{lstlisting}

\textbf{Входные параметры:}
 
VMHL\_MatrixXY --- указатель на матрицу значений X и Y графиков;
 
VMHL\_N\_EveryCol --- количество элементов в каждом столбце (так как столбцы идут по парам, то число элементов в нечетном и
 
следующем за ним четном столбце должны совпадать, например 10,10,5,5,7,7);
 
VMHL\_M --- количество столбцов матрицы (должно быть четным числом конечно);
 
TitleChart --- заголовок графика;
 
NameVectorX --- название оси Ox. В формате: [обозначение], [расшифровка]. Например: u, Вероятность выбора;
 
NameVectorY --- название оси Oy. В формате: [обозначение], [расшифровка]. Например: q, Количество абрикосов;
 
NameLine --- указатель на вектор названий графиков (для легенды) количество элементов VMHL\_M/2;
 
Label --- label для графика;
 
ShowLine --- показывать ли линию;
 
ShowPoints --- показывать ли точки;
 
ShowArea --- показывать ли закрашенную область под кривой;
 
ShowSpecPoints --- показывать ли специальные точки;
 
ForNormalSize --- нормальный размер графика (на всю ширину) или для маленького размера график создается;
 
GrayStyle --- серый стиль графиков;
 
SolidStyle --- линии делать сплошными или разными по типу (точками, тире и др.);
 
CircleStyle --- точки все делать кругляшками или нет.
	
\textbf{Возвращаемое значение:}

Строка с Latex кодами с выводимым графиком.


\subsubsection{THQt\_LatexShowTwoChartsOfLine}\label{THQt_LatexShowTwoChartsOfLine}

Функция возвращает строку с выводом некоторых двух графиков по точкам с Latex кодами.


\begin{lstlisting}[label=code_syntax_THQt_LatexShowTwoChartsOfLine,caption=Синтаксис]
template <class T> QString THQt_LatexShowTwoChartsOfLine (T *VMHL_VectorX,T *VMHL_VectorY1,T *VMHL_VectorY2, int VMHL_N, QString TitleChart, QString NameVectorX, QString NameVectorY,QString NameLine1, QString NameLine2, QString Label,bool ShowLine,bool ShowPoints,bool ShowArea,bool ShowSpecPoints, bool ForNormalSize, bool GrayStyle);
template <class T> QString THQt_LatexShowTwoChartsOfLine (T *VMHL_VectorX,T *VMHL_VectorY1,T *VMHL_VectorY2, int VMHL_N, QString TitleChart, QString NameVectorX, QString NameVectorY,QString NameLine1, QString NameLine2, QString Label,bool ShowLine,bool ShowPoints,bool ShowArea,bool ShowSpecPoints, bool ForNormalSize);
template <class T> QString THQt_LatexShowTwoChartsOfLine (T *VMHL_VectorX,T *VMHL_VectorY1,T *VMHL_VectorY2, int VMHL_N, QString TitleChart, QString NameVectorX, QString NameVectorY,QString NameLine1, QString NameLine2, QString Label,bool ShowLine,bool ShowPoints,bool ShowArea,bool ShowSpecPoints);
template <class T> QString THQt_LatexShowTwoChartsOfLine (T *VMHL_VectorX,T *VMHL_VectorY1,T *VMHL_VectorY2, int VMHL_N, QString TitleChart, QString NameVectorX, QString NameVectorY,QString NameLine1, QString NameLine2, QString Label);
template <class T> QString THQt_LatexShowTwoChartsOfLine (T *VMHL_VectorX,T *VMHL_VectorY1,T *VMHL_VectorY2, int VMHL_N);
\end{lstlisting}

\textbf{Входные параметры:}
 
    VMHL\_VectorX --- указатель на вектор координат X точек;
 
    VMHL\_VectorY1 --- указатель на вектор координат Y точек первой линии;
 
    VMHL\_VectorY2 --- указатель на вектор координат Y точек второй линии;
 
    VMHL\_N --- количество точек;
 
    TitleChart --- заголовок графика;
 
    NameVectorX --- название оси Ox. В формате: [обозначение], [расшифровка]. Например: u, Вероятность выбора;
 
    NameVectorY --- название оси Oy. В формате: [обозначение], [расшифровка]. Например: q, Количество абрикосов;
 
    NameLine1 --- название первого графика (для легенды);
 
    NameLine2 --- название второго графика (для легенды);
 
    Label --- label для графика;
 
    ShowLine --- показывать ли линию;
 
    ShowPoints --- показывать ли точки;
 
    ShowArea --- показывать ли закрашенную область под кривой;
 
    ShowSpecPoints --- показывать ли специальные точки;
 
    ForNormalSize --- нормальный размер графика (на всю ширину) или для маленького размера график создается;
 
    GrayStyle --- второй график рисовать серым, а не красным.
	
\textbf{Возвращаемое значение:}

Строка с Latex кодами с выводимым графиком.


\subsubsection{THQt\_LatexShowTwoIndependentChartsOfLine}\label{THQt_LatexShowTwoIndependentChartsOfLine}

Функция возвращает строку с выводом некоторых двух независимых графиков по точкам с Latex кодами.


\begin{lstlisting}[label=code_syntax_THQt_LatexShowTwoIndependentChartsOfLine,caption=Синтаксис]
template <class T> QString THQt_LatexShowTwoIndependentChartsOfLine (T *VMHL_VectorX1,T *VMHL_VectorY1,int VMHL_N1,T *VMHL_VectorX2,T *VMHL_VectorY2, int VMHL_N2, QString TitleChart, QString NameVectorX, QString NameVectorY,QString NameLine1, QString NameLine2, QString Label, bool ShowLine,bool ShowPoints,bool ShowArea,bool ShowSpecPoints, bool ForNormalSize, bool GrayStyle);
template <class T> QString THQt_LatexShowTwoIndependentChartsOfLine (T *VMHL_VectorX1,T *VMHL_VectorY1,int VMHL_N1,T *VMHL_VectorX2,T *VMHL_VectorY2, int VMHL_N2, QString TitleChart, QString NameVectorX, QString NameVectorY,QString NameLine1, QString NameLine2, QString Label, bool ShowLine,bool ShowPoints,bool ShowArea,bool ShowSpecPoints, bool ForNormalSize);
template <class T> QString THQt_LatexShowTwoIndependentChartsOfLine (T *VMHL_VectorX1,T *VMHL_VectorY1,int VMHL_N1,T *VMHL_VectorX2,T *VMHL_VectorY2, int VMHL_N2, QString TitleChart, QString NameVectorX, QString NameVectorY,QString NameLine1, QString NameLine2, QString Label, bool ShowLine,bool ShowPoints,bool ShowArea,bool ShowSpecPoints);
template <class T> QString THQt_LatexShowTwoIndependentChartsOfLine (T *VMHL_VectorX1,T *VMHL_VectorY1,int VMHL_N1,T *VMHL_VectorX2,T *VMHL_VectorY2, int VMHL_N2, QString TitleChart, QString NameVectorX, QString NameVectorY,QString NameLine1, QString NameLine2, QString Label);
template <class T> QString THQt_LatexShowTwoIndependentChartsOfLine (T *VMHL_VectorX1,T *VMHL_VectorY1,int VMHL_N1,T *VMHL_VectorX2,T *VMHL_VectorY2, int VMHL_N2);
\end{lstlisting}

\textbf{Входные параметры:}
 
    VMHL\_VectorX1 --- указатель на вектор координат X точек первой линии;
 
    VMHL\_VectorY1 --- указатель на вектор координат Y точек первой линии;
 
    VMHL\_N1 --- количество точек первой линии;
 
    VMHL\_VectorX2 --- указатель на вектор координат X точек второй линии;
 
    VMHL\_VectorY2 --- указатель на вектор координат Y точек второй линии;
 
    VMHL\_N2 --- количество точек второй линии;
 
    TitleChart --- заголовок графика;
 
    NameVectorX --- название оси Ox. В формате: [обозначение], [расшифровка]. Например: u, Вероятность выбора;
 
    NameVectorY --- название оси Oy. В формате: [обозначение], [расшифровка]. Например: q, Количество абрикосов;
 
    NameLine1 --- название первого графика (для легенды);
 
    NameLine2 --- название второго графика (для легенды);
 
    Label --- label для графика;
 
    ShowLine --- показывать ли линию;
 
    ShowPoints --- показывать ли точки;
 
    ShowArea --- показывать ли закрашенную область под кривой;
 
    ShowSpecPoints --- показывать ли специальные точки;
 
    ForNormalSize --- нормальный размер графика (на всю ширину) или для маленького размера график создается;
 
    GrayStyle --- второй график рисовать серым, а не красным.
	
\textbf{Возвращаемое значение:}

Строка с Latex кодами с выводимым графиком.


\subsubsection{THQt\_LatexShowTwoIndependentChartsOfPointsAndLine}\label{THQt_LatexShowTwoIndependentChartsOfPointsAndLine}

Функция возвращает строку с выводом некоторого двух независимых графиков по точкам с Latex кодами. Один график выводится в виде точек, а второй в виде линии. Удобно для отображения регрессий.


\begin{lstlisting}[label=code_syntax_THQt_LatexShowTwoIndependentChartsOfPointsAndLine,caption=Синтаксис]
template <class T> QString THQt_LatexShowTwoIndependentChartsOfPointsAndLine (T *VMHL_VectorX1,T *VMHL_VectorY1,int VMHL_N1,T *VMHL_VectorX2,T *VMHL_VectorY2, int VMHL_N2, QString TitleChart, QString NameVectorX, QString NameVectorY,QString NameLine1, QString NameLine2, QString Label,bool ShowLine,bool ShowPoints,bool ShowArea,bool ShowSpecPoints, bool ForNormalSize, bool GrayStyle);
template <class T> QString THQt_LatexShowTwoIndependentChartsOfPointsAndLine (T *VMHL_VectorX1,T *VMHL_VectorY1,int VMHL_N1,T *VMHL_VectorX2,T *VMHL_VectorY2, int VMHL_N2, QString TitleChart, QString NameVectorX, QString NameVectorY,QString NameLine1, QString NameLine2, QString Label,bool ShowLine,bool ShowPoints,bool ShowArea,bool ShowSpecPoints, bool ForNormalSize);
template <class T> QString THQt_LatexShowTwoIndependentChartsOfPointsAndLine (T *VMHL_VectorX1,T *VMHL_VectorY1,int VMHL_N1,T *VMHL_VectorX2,T *VMHL_VectorY2, int VMHL_N2, QString TitleChart, QString NameVectorX, QString NameVectorY,QString NameLine1, QString NameLine2, QString Label,bool ShowLine,bool ShowPoints,bool ShowArea,bool ShowSpecPoints);
template <class T> QString THQt_LatexShowTwoIndependentChartsOfPointsAndLine (T *VMHL_VectorX1,T *VMHL_VectorY1,int VMHL_N1,T *VMHL_VectorX2,T *VMHL_VectorY2, int VMHL_N2, QString TitleChart, QString NameVectorX, QString NameVectorY,QString NameLine1, QString NameLine2, QString Label);
template <class T> QString THQt_LatexShowTwoIndependentChartsOfPointsAndLine (T *VMHL_VectorX1,T *VMHL_VectorY1,int VMHL_N1,T *VMHL_VectorX2,T *VMHL_VectorY2, int VMHL_N2);
\end{lstlisting}

\textbf{Входные параметры:}
 
    VMHL\_VectorX1 --- указатель на вектор координат X точек первой линии;
 
    VMHL\_VectorY1 --- указатель на вектор координат Y точек первой линии;
 
    VMHL\_N1 --- количество точек первой линии;
 
    VMHL\_VectorX2 --- указатель на вектор координат X точек второй линии;
 
    VMHL\_VectorY2 --- указатель на вектор координат Y точек второй линии;
 
    VMHL\_N2 --- количество точек второй линии;
 
    TitleChart --- заголовок графика;
 
    NameVectorX --- название оси Ox. В формате: [обозначение], [расшифровка]. Например: u, Вероятность выбора;
 
    NameVectorY --- название оси Oy. В формате: [обозначение], [расшифровка]. Например: q, Количество абрикосов;
 
    NameLine1 --- название первого графика (для легенды);
 
    NameLine2 --- название второго графика (для легенды);
 
    Label --- label для графика;
 
    ShowLine --- показывать ли линию;
 
    ShowPoints --- показывать ли точки;
 
    ShowArea --- показывать ли закрашенную область под кривой;
 
    ShowSpecPoints --- показывать ли специальные точки;
 
    ForNormalSize --- нормальный размер графика (на всю ширину) или для маленького размера график создается;
 
    GrayStyle --- второй график рисовать серым, а не красным.
	
\textbf{Возвращаемое значение:}

Строка с Latex кодами с выводимым графиком.


\subsection{Обработка текста}

\subsubsection{HQt\_ForcedWordWrap}\label{HQt_ForcedWordWrap}

Функция расставляет принудительные переносы в стиле Latex.


\begin{lstlisting}[label=code_syntax_HQt_ForcedWordWrap,caption=Синтаксис]
QString HQt_ForcedWordWrap(QString S);
\end{lstlisting}

\textbf{Входные параметры:}

S --- разбиваемая строка.

\textbf{Возвращаемое значение:}

 Срока с расставленными принудительно переносами.


\subsubsection{HQt\_LatexGreenText}\label{HQt_LatexGreenText}

Функция возвращает строку с выводом зеленого текста.


\begin{lstlisting}[label=code_syntax_HQt_LatexGreenText,caption=Синтаксис]
QString HQt_LatexGreenText (QString String);
\end{lstlisting}

\textbf{Входные параметры:}

String --- непосредственно выводимая строка.

\textbf{Возвращаемое значение:}

Строка с Latex кодами с зеленым текстом.


\subsubsection{HQt\_LatexRedText}\label{HQt_LatexRedText}

Функция возвращает строку с выводом красного текста.


\begin{lstlisting}[label=code_syntax_HQt_LatexRedText,caption=Синтаксис]
QString HQt_LatexRedText (QString String);
\end{lstlisting}

\textbf{Входные параметры:}

String --- непосредственно выводимая строка.

\textbf{Возвращаемое значение:}

Строка с Latex кодами с красным текстом.


\subsubsection{HQt\_TextForLatexToText}\label{HQt_TextForLatexToText}

Функция обрабатывает строку String из переделки функции HQt\_TextToTextForLatex в нормальную строку. Еще удаляются знаки \$, которые обрамляют формулы.


\begin{lstlisting}[label=code_syntax_HQt_TextForLatexToText,caption=Синтаксис]
QString HQt_TextForLatexToText (QString String);
\end{lstlisting}

\textbf{Входные параметры:}

String --- обрабатываемая строка.

\textbf{Возвращаемое значение:}
 
Обработанная строка.


\subsubsection{HQt\_TextToTextForLatex}\label{HQt_TextToTextForLatex}

Функция переводит текст в текст, который можно добавить в Latex код. В-первую очередь, это экранирование некоторых элементов.


\begin{lstlisting}[label=code_syntax_HQt_TextToTextForLatex,caption=Синтаксис]
QString HQt_TextToTextForLatex (QString Text);
\end{lstlisting}

\textbf{Входные параметры:}

TitleX --- непосредственно выводимая строка.

\textbf{Возвращаемое значение:}

Измененный текст, который можно добавлять в LaTeX.


\subsubsection{THQt\_LatexNumberToText}\label{THQt_LatexNumberToText}

Функция выводит число VMHL\_X в строку Latex, причем число выделено жирным.


\begin{lstlisting}[label=code_syntax_THQt_LatexNumberToText,caption=Синтаксис]
template <class T> QString THQt_LatexNumberToText (T VMHL_X);
\end{lstlisting}

\textbf{Входные параметры:}

VMHL\_X --- выводимое число.

\textbf{Возвращаемое значение:}

Строка, в которой записано число.


\subsection{Показ математических выражений}

\subsubsection{THQt\_LatexShowMatrix}\label{THQt_LatexShowMatrix}

Функция возвращает строку с выводом некоторой матрицы VMHL\_Matrix с Latex кодами.


\begin{lstlisting}[label=code_syntax_THQt_LatexShowMatrix,caption=Синтаксис]
template <class T> QString THQt_LatexShowMatrix (T *VMHL_Matrix, int VMHL_N, int VMHL_M, QString TitleMatrix, QString NameMatrix);
template <class T> QString THQt_LatexShowMatrix (T *VMHL_Matrix, int VMHL_N, int VMHL_M, QString NameMatrix);
template <class T> QString THQt_LatexShowMatrix (T *VMHL_Matrix, int VMHL_N, int VMHL_M);
\end{lstlisting}

\textbf{Входные параметры:}

VMHL\_Matrix --- указатель на выводимую матрицу;
 
VMHL\_N --- количество строк в матрице;
 
VMHL\_M --- количество столбцов в матрице;
 
TitleMatrix --- заголовок выводимой матрицы;
 
NameMatrix --- обозначение матрицы.
	
\textbf{Возвращаемое значение:}

Строка с Latex кодами с выводимой матрицей.


\subsubsection{THQt\_LatexShowMatrix2}\label{THQt_LatexShowMatrix2}

Функция возвращает строку с выводом некоторой матрицы VMHL\_Matrix с Latex кодами.


\begin{lstlisting}[label=code_syntax_THQt_LatexShowMatrix2,caption=Синтаксис]
QString THQt_LatexShowMatrix (QStringList *VMHL_Matrix, int VMHL_N, QString TitleMatrix, QString NameMatrix);
QString THQt_LatexShowMatrix (QStringList *VMHL_Matrix, int VMHL_N, QString NameMatrix);
QString THQt_LatexShowMatrix (QStringList *VMHL_Matrix, int VMHL_N);
\end{lstlisting}

\textbf{Входные параметры:}

    VMHL\_Matrix --- указатель на выводимую матрицу;
 
    VMHL\_N --- количество строк в матрице;
 
    TitleMatrix --- заголовок выводимой матрицы;
 
    NameMatrix --- обозначение матрицы.
	
\textbf{Возвращаемое значение:}

Строка с Latex кодами с выводимой матрицей.


\subsubsection{THQt\_LatexShowVector}\label{THQt_LatexShowVector}

Функция возвращает строку с выводом некоторого вектора VMHL\_Vector с Latex кодами.


\begin{lstlisting}[label=code_syntax_THQt_LatexShowVector,caption=Синтаксис]
template <class T> QString THQt_LatexShowVector (T *VMHL_Vector, int VMHL_N, QString TitleVector, QString NameVector);
template <class T> QString THQt_LatexShowVector (T *VMHL_Vector, int VMHL_N, QString NameVector);
template <class T> QString THQt_LatexShowVector (T *VMHL_Vector, int VMHL_N);
\end{lstlisting}

\textbf{Входные параметры:}
 
    VMHL\_Vector --- указатель на выводимый вектор;
 
    VMHL\_N --- количество элементов вектора;
 
    TitleVector --- заголовок выводимого вектора;
 
    NameVector --- обозначение вектора.
	
\textbf{Возвращаемое значение:}

Строка с Latex кодами с выводимым вектором.


\subsubsection{THQt\_LatexShowVector2}\label{THQt_LatexShowVector2}

Функция возвращает строку с выводом некоторого вектора VMH\_Vector с Latex кодами.


\begin{lstlisting}[label=code_syntax_THQt_LatexShowVector2,caption=Синтаксис]
QString THQt_LatexShowVector (QStringList VMHL_Vector, QString TitleVector, QString NameVector);
QString THQt_LatexShowVector (QStringList VMHL_Vector, QString NameVector);
QString THQt_LatexShowVector (QStringList VMHL_Vector);
\end{lstlisting}

\textbf{Входные параметры:}
 
    VMHL\_Vector --- указатель на выводимый вектор;
 
    TitleVector --- заголовок выводимого вектора;
 
    NameVector --- обозначение вектора.
	
\textbf{Возвращаемое значение:}

Строка с Latex кодами с выводимым вектором.


\subsubsection{THQt\_LatexShowVectorT}\label{THQt_LatexShowVectorT}

Функция возвращает строку с выводом некоторого вектора VMHL\_Vector в транспонированном виде с Latex кодами.


\begin{lstlisting}[label=code_syntax_THQt_LatexShowVectorT,caption=Синтаксис]
template <class T> QString THQt_LatexShowVectorT (T *VMHL_Vector, int VMHL_N, QString TitleVector, QString NameVector);
template <class T> QString THQt_LatexShowVectorT (T *VMHL_Vector, int VMHL_N, QString NameVector);
template <class T> QString THQt_LatexShowVectorT (T *VMHL_Vector, int VMHL_N);
\end{lstlisting}

\textbf{Входные параметры:}
 
    VMHL\_Vector --- указатель на выводимый вектор;
 
    VMHL\_N --- количество элементов вектора;
 
    TitleVector --- заголовок выводимого вектора;
 
    NameVector --- обозначение вектора.
	
\textbf{Возвращаемое значение:}

Строка с Latex кодами с выводимым вектором.


\subsection{Составные изображения}

\subsubsection{HQt\_LatexBeginCompositionFigure}\label{HQt_LatexBeginCompositionFigure}

Функция возвращает строку с выводом начала рисунка, состоящего из нескольких рисунков или графиков.


\begin{lstlisting}[label=code_syntax_HQt_LatexBeginCompositionFigure,caption=Синтаксис]
QString HQt_LatexBeginCompositionFigure ();
\end{lstlisting}

\textbf{Входные параметры:}

Отсутствует.

\textbf{Возвращаемое значение:}

Строка с Latex кодами.


\begin{lstlisting}[label=code_use_HQt_LatexBeginCompositionFigure,caption=Пример использования]
Latex += HQt_LatexBeginCompositionFigure ();
Latex += HQt_LatexBeginFigureInCompositionFigure ();
Latex += THQt_LatexShowChartOfLine (dataX,dataY,N,"Тестовый график","u, Вероятность выбора","q, Количество воронов","линия","plot1",true,true,true,true,false,false);
Latex += HQt_LatexEndFigureInCompositionFigure ();
Latex += HQt_LatexBeginFigureInCompositionFigure ();
Latex += THQt_LatexShowChartOfLine (dataX,dataY,N,"Тестовый график","u, Вероятность выбора","q, Количество воронов","линия","plot2",true,true,true,true,false,false);
Latex += HQt_LatexEndFigureInCompositionFigure ();
Latex += HQt_LatexEndCompositionFigure ("Два графика", "TwoFig");
\end{lstlisting}

\subsubsection{HQt\_LatexBeginFigureInCompositionFigure}\label{HQt_LatexBeginFigureInCompositionFigure}

Функция возвращает строку с Latex кодом при добавлении дополнительного рисунка или графика в рисунок, состоящего из нескольких рисунков.


\begin{lstlisting}[label=code_syntax_HQt_LatexBeginFigureInCompositionFigure,caption=Синтаксис]
QString HQt_LatexBeginFigureInCompositionFigure ();
\end{lstlisting}

\textbf{Входные параметры:}

Отсутствует.

\textbf{Возвращаемое значение:}

Строка с Latex кодами.


\subsubsection{HQt\_LatexEndCompositionFigure}\label{HQt_LatexEndCompositionFigure}

Функция возвращает строку с выводом окончания рисунка, состоящего из нескольких рисунков или графиков.


\begin{lstlisting}[label=code_syntax_HQt_LatexEndCompositionFigure,caption=Синтаксис]
QString HQt_LatexEndCompositionFigure (QString TitleFigure, QString Label);
QString HQt_LatexEndCompositionFigure (QString TitleFigure);
QString HQt_LatexEndCompositionFigure ();
\end{lstlisting}

\textbf{Входные параметры:}

TitleFigure --- заголовок рисунка;

     Label --- label для рисунка.

\textbf{Возвращаемое значение:}

Строка с Latex кодами.


\begin{lstlisting}[label=code_use_HQt_LatexEndCompositionFigure,caption=Пример использования]
Latex += HQt_LatexBeginCompositionFigure ();
Latex += HQt_LatexBeginFigureInCompositionFigure ();
Latex += THQt_LatexShowChartOfLine (dataX,dataY,N,"Тестовый график","u, Вероятность выбора","q, Количество воронов","линия","plot1",true,true,true,true,false,false);
Latex += HQt_LatexEndFigureInCompositionFigure ();
Latex += HQt_LatexBeginFigureInCompositionFigure ();
Latex += THQt_LatexShowChartOfLine (dataX,dataY,N,"Тестовый график","u, Вероятность выбора","q, Количество воронов","линия","plot2",true,true,true,true,false,false);
Latex += HQt_LatexEndFigureInCompositionFigure ();
Latex += HQt_LatexEndCompositionFigure ("Два графика", "TwoFig");
\end{lstlisting}

\subsubsection{HQt\_LatexEndFigureInCompositionFigure}\label{HQt_LatexEndFigureInCompositionFigure}

Функция возвращает строку с Latex кодом после добавлении дополнительного рисунка или графика в рисунок, состоящего из нескольких рисунков.


\begin{lstlisting}[label=code_syntax_HQt_LatexEndFigureInCompositionFigure,caption=Синтаксис]
QString HQt_LatexEndFigureInCompositionFigure ();
\end{lstlisting}

\textbf{Входные параметры:}

Отсутствует.

\textbf{Возвращаемое значение:}

Строка с Latex кодами.


\subsection{Таблицы}

\subsubsection{HQt\_LatexShowTable}\label{HQt_LatexShowTable}

Функция возвращает строку с выводом таблицы.


\begin{lstlisting}[label=code_syntax_HQt_LatexShowTable,caption=Синтаксис]
QString HQt_LatexShowTable (QStringList Col1, QStringList Col2, QString NameCol1, QString NameCol2, double WidthCol1, QString Title);
QString HQt_LatexShowTable (QStringList Col1, QStringList Col2, QStringList Col3, QString NameCol1, QString NameCol2, QString NameCol3, double WidthCol1, double WidthCol2, QString Title);
\end{lstlisting}

\textbf{Входные параметры:}
 
    Col1 --- список строк первого столбца;
 
    Col2 --- список строк второго столбца;
 
    NameCol1--- заголовок первого столбца;
 
    NameCol2--- заголовок второго столбца;
 
    WidthCol1 --- ширина первого столбца в процентах, например 50%;
 
    Title --- заголовок таблицы.

\textbf{Возвращаемое значение:}

Строка с Latex кодами с выводимой таблицы.


\subsection{Текст}

\subsubsection{HQt\_LatexShowAlert}\label{HQt_LatexShowAlert}

Функция возвращает строку с выводом некоторого предупреждения.


\begin{lstlisting}[label=code_syntax_HQt_LatexShowAlert,caption=Синтаксис]
QString HQt_LatexShowAlert (QString String);
\end{lstlisting}

\textbf{Входные параметры:}

String --- непосредственно выводимая строка.

\textbf{Возвращаемое значение:}

Строка с Latex кодами с выводимым предупреждением.


\subsubsection{HQt\_LatexShowHr}\label{HQt_LatexShowHr}

Функция возвращает строку с выводом горизонтальной линии.


\begin{lstlisting}[label=code_syntax_HQt_LatexShowHr,caption=Синтаксис]
QString HQt_LatexShowHr ();
\end{lstlisting}

\textbf{Входные параметры:}

Отсутствуют.

\textbf{Возвращаемое значение:}

Строка с Latex кодами с тэгом горизонтальной линии.


\subsubsection{HQt\_LatexShowSection}\label{HQt_LatexShowSection}

Функция возвращает строку с выводом некоторой строки в виде заголовка.


\begin{lstlisting}[label=code_syntax_HQt_LatexShowSection,caption=Синтаксис]
QString HQt_LatexShowSection (QString String);
\end{lstlisting}

\textbf{Входные параметры:}

String --- непосредственно выводимая строка.

\textbf{Возвращаемое значение:}

Строка с Latex кодами с выводимой строкой.


\subsubsection{HQt\_LatexShowSimpleText}\label{HQt_LatexShowSimpleText}

Функция возвращает строку с выводом некоторой строки с Latex кодами без всякого излишества.


\begin{lstlisting}[label=code_syntax_HQt_LatexShowSimpleText,caption=Синтаксис]
QString HQt_LatexShowSimpleText (QString String);
\end{lstlisting}

\textbf{Входные параметры:}

String --- непосредственно выводимая строка.

\textbf{Возвращаемое значение:}

Строка с Latex кодами с выводимой строкой.


\subsubsection{HQt\_LatexShowSubsection}\label{HQt_LatexShowSubsection}

Функция возвращает строку с выводом некоторой строки в виде подзаголовка.


\begin{lstlisting}[label=code_syntax_HQt_LatexShowSubsection,caption=Синтаксис]
QString HQt_LatexShowSubsection (QString String);
\end{lstlisting}

\textbf{Входные параметры:}

String --- непосредственно выводимая строка.

\textbf{Возвращаемое значение:}

Строка с Latex кодами с выводимой строкой.


\subsubsection{HQt\_LatexShowText}\label{HQt_LatexShowText}

Функция возвращает строку с выводом некоторой строки с Latex кодами.


\begin{lstlisting}[label=code_syntax_HQt_LatexShowText,caption=Синтаксис]
QString HQt_LatexShowText (QString TitleX);
\end{lstlisting}

\textbf{Входные параметры:}

TitleX --- непосредственно выводимая строка.

\textbf{Возвращаемое значение:}

Строка с Latex кодами с выводимой строкой (в виде абзаца).


\subsubsection{THQt\_LatexShowNumber}\label{THQt_LatexShowNumber}

Функция возвращает строку с выводом некоторого числа VMHL\_X с Latex кодами.


\begin{lstlisting}[label=code_syntax_THQt_LatexShowNumber,caption=Синтаксис]
template <class T> QString THQt_LatexShowNumber (T VMHL_X, QString TitleX, QString NameX);
template <class T> QString THQt_LatexShowNumber (T VMHL_X, QString NameX);
template <class T> QString THQt_LatexShowNumber (T VMHL_X);
\end{lstlisting}

\textbf{Входные параметры:}
 
VMHL\_X --- выводимое число;
 
TitleX --- заголовок выводимого числа;
 
NameX --- обозначение числа.

\textbf{Возвращаемое значение:}

Строка с Latex кодами с выводимым числом.

%%%%%%%%%%%%%%%%%%%%%%%%%%%%%%%%%%%%%%%%%%%%%%%%%%%%%%%%%%

\end{document}